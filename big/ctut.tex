\documentclass[a4paper,10pt]{report}

%% \documentclass[a4paper,10pt,twocolumn]{article}
\usepackage{titling}
\title{An Introduction to Programming in C}

\author{David Morris}

\usepackage[cm]{fullpage}
\usepackage[small,compact]{titlesec}

%% \usepackage{fancyhdr}
%% \fancypagestyle{fancyplain}{
%%   \fancyhf{}
%%   \fancyhead[L]{\leftmark}
%%   \fancyfoot[R]{\thepage}
%%   \fancyfoot[L]{\theauthor}
%%   \setlength{\headheight}{15pt}
%%   \setlength{\headsep}{15pt}
%% }
%% \pagestyle{fancyplain}

%% \fancypagestyle{plain}{
%%   \fancyhf{}
%%   \fancyfoot[R]{\thepage}
%%   \setlength{\headheight}{0pt}
%%   \renewcommand{\headrulewidth}{0pt}
%% }

\usepackage{listings}
\usepackage{courier}
\usepackage{bashful}

\lstset{
  basicstyle=\ttfamily,
  showstringspaces=false,
  breakatwhitespace=true,
  nolol=true
}

\lstdefinestyle{drmCodeStyle}{frame=lines}

\newcommand{\drmcode}[3]{\begin{minipage}{\columnwidth}\lstinputlisting[language={#1},style=drmCodeStyle,caption={#2},name={},nolol=false]{#3}\end{minipage}}
\newcommand{\ccode}[2]{\drmcode{C}{#1}{#2.c}}
\newcommand{\largeccode}[2]{\lstinputlisting[language=C,style=drmCodeStyle,caption={#2},name={},nolol=false]{#2.c}}
\newcommand{\perlcode}[2]{\drmcode{Perl}{#1}{#2.pl}}
\newcommand{\makecode}[2]{\drmcode{Make}{#1}{#2}}
\newcommand{\plaincode}[2]{\drmcode{}{#1}{#2}}

%% \bash
%% ./kwgen.pl keywords.txt > keywords.tex
%% \END

%% \newcommand{\toolgcc}{\textt{gcc}}
\newcommand{\toolmake}{\textt{make}}
\newcommand{\toolnm}{\textt{nm}}
\newcommand{\toolar}{\textt{ar}}
\newcommand{\toolld}{\textt{ld}}
\newcommand{\toolgdb}{\textt{gdb}}
\newcommand{\kwstatic}{\textt{static}}
\newcommand{\typeint}{\textt{int}}
\newcommand{\typeunsigned}{\textt{unsigned}}
\newcommand{\typeshort}{\textt{short}}


\newcommand{\kw}[1]{\texttt{#1}}

\begin{document}

\maketitle

\tableofcontents

\lstlistoflistings

\part{Basics}

\chapter{Introduction}

This document is an introduction to programming using the C language
and ecosystem. It assumes that the reader is previously familiar with
one or more programming languages, and will not appreciate yet another
introduction to \texttt{if} statements and the strange and wonderful
concept of a `variable'.

\section{Conventions used in this document}

\subsection{Terminal Interaction}

We will make extensive use of a \textsc{Unix} or *NIX
shell. \texttt{Bash} is used throughout; it is assumed that anyone
running anything more exotic is capable of dealing with the
differences.

Shell commands and output are presented as follows:

\bash[script,stdout]
echo "Hello"
\END

\noindent
In other words, an instruction to \emph{run} a command is presented
as:

\bash[script]
echo "Hello"
\END

\noindent
note that the \texttt{\%} sign signifies a shell prompt, and should
not be typed into the terminal. Command output is presented as:

\bash[stdout]
echo "Hello"
\END

\noindent
In almost all cases, 

\section{Hello World}

We should begin by confirming that the C programming environment on
your machine is reasonably sane. For this, we use the traditional
``Hello, World'' program. Enter the following program into the text
editor of your choice, and save it as ``hello.c''.

\ccode{\texttt{hello.c}}{hello}

\noindent
Enter the following command into your terminal to compile
the program:

\bash[script,stdout]
gcc -o hello hello.c
\END

\noindent
and finally run the program with:

\bash[script, stdout]
./hello
\END

\noindent{}
If you get error messages, or the output does not appear as shown,
there is little point continuing until these issues have been
resolved. It is assumed that anyone reading this document has
sufficient intelligence and network bandwidth to resolve problems here
on their own.

\chapter{Compilation}

Rather than looking at particular pieces of the language ---
functions, \kw{if} statements, loops, variables --- let's look at how
we get from a C program in text file to a running binary. Many aspects
of C result from this process, so rather than waving our hands and
ascribing these to some kind of digital voodoo, we'll see the
\emph{reasons} that things are the way they are. Then hopefully the
pieces themselves should follow fairly simply.

\section{Preprocessing}

\subsection{Introduction}

The first stage of C compilation is preprocessing. The preprocessor is
the tool that handles the \verb!#include<stdio.h>! line of our
``Hello, World'' program. Usually, this line is explained to those new
to C as `use the \kw{stdio} library, `import the \kw{stdio} module',
or something like that.

What it actually means is something like ``replace this line with the
entire contents of the \kw{stdio.h} file''. We will come in a moment
to how the preprocessor finds \kw{stdio.h}, but the key thing is that
a \verb!#include! line is \emph{purely textual} --- it does not know
anything about modules, or libraries or any of the `semantic' concepts
we use to talk about program structure.

This is true of the preprocessor in general --- when we use the
preprocessor, we are talking about characters, and lines, and files,
\emph{not} any kind of larger \emph{conceptual} building block. For
this reason, we usually try to avoid using the preprocessor when
there's another way to do something, because throwing characters
around is about the least-reliable and hardest-to-understand way of
doing things.

However, it's pretty much impossible to write a C program that does
anything useful without using the preprocessor, so you need to
understand it.

\subsection{Invoking the Preprocessor}

Normally, the preprocessor is invoked implicitly by the compiler ---
when we executed \verb!gcc -o hello hello.c! earlier, \kw{gcc}
automatically ran the preprocessor (along with all of the other stages
of the compilation).

However, if we want to examine the preprocessor in more detail, we
need a way of seeing what it does to a file on its own.

\kw{gcc} can be told to output the result of preprocessing using the
\kw{-E} flag. Try running it on your \kw{hello.c} file from earlier:

\bash[script]
gcc -E hello.c
\END

This prints lots of lines (most of them blank) to your terminal. Most
of them aren't very interesting (you can find the whole lot --- at
least for my machine --- in Appendix \ref{cpphello}).

There are a few interesting points about this output. 

\subsection{Simple Macros}

\subsection{Conditional Compilation}

\subsection{More Macros}

\subsection{Dependencies}

\section{Compilation}

After preprocessing, the compiler gets a stream of text in the `real'
C language.

\section{Assembly}



\section{Linking}

\subsection{Static Libraries}

\subsection{\kw{nm}}

\kw{nm} is a tool for looking at the symbols in an object file,
archive, or shared library.

\subsection{Dynamic Libraries}

\section{\kw{make}}

\kw{make} is a tool which automates the building (that is,
transformation from source code to executable) of large C programs. In
particular, it makes it easier to do reliable \emph{incremental}
builds --- that is, builds in which only \emph{some} of the C source
files have changed.

As we have seen, each source file (or translation unit) can be
independently converted into an object file. For large projects ---
with large numbers of source files --- this individual source file
translation is the lion's share of the time taken for the build.

However, we only need to perform this translation when a given source
file has changed. This gives us an opportunity to massively speed up
the build process, by keeping the object files from previous builds,
and only creating new ones as necessary.

To do this, we need to keep track of which source files have been
changed since the last build, and which source files were 

\chapter{Memory}

\section{The C Memory Model}

\subsection{Pointers}

\subsection{Arrays}

\subsection{Allocation}

\chapter{Debugging}

\subsection{\kw{printf}}

\subsection{\kw{assert}}

\subsection{\kw{gdb}}

\chapter{Modules}

\section{Single-Instance Modules}

\subsection{Implementing \kw{strtok}}

\section{Multiple-Instance Modules}

\subsection{Multiple Instances of \kw{strtok}}

\section{Dynamic Interfaces}

\section{Polymorphic Interfaces}

\part{Worked Examples}

\part{Reference}

\chapter{Syntax}

\section{Built-in Types}

\begin{description}
  \item[\kw{int}]
  \item[\kw{char}]
  \item[\kw{short}]
  \item[\kw{long}]
  \item[\kw{long long}]
  \item[\kw{float}]
  \item[\kw{double}]
\end{description}

\section{Other Common Types}

\begin{description}
  \item[\kw{FILE}]
  \item[\kw{size\_t}]
  \item[\kw{time\_t}]
\end{description}

\section{Variables}

\section{Control Flow}

\subsection{\kw{if}}

\subsection{\kw{switch}}

\subsection{\kw{while}}

\subsection{\kw{do}/\kw{while}}

\subsection{\kw{for}}

\section{Arithmetic Operators}

\section{Bitwise Operators}

\chapter{The Standard Library}

\part{Utilities}

\part{Appendices}

\appendix

\chapter{Output of \texttt{gcc -E hello.c}}
\label{cpphello}

\bash[script,stdout]
gcc -E hello.c
\END


\end{document}
