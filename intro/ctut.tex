\documentclass[a4paper,10pt]{article}

\usepackage{titling}
\title{An Introduction to Programming in C}

\author{David Morris}

\usepackage[cm]{fullpage}
\usepackage[small,compact]{titlesec}

\usepackage{fancyhdr}
\fancypagestyle{fancyplain}{
  \fancyhf{}
  \fancyhead[L]{\leftmark}
  \fancyfoot[R]{\thepage}
  \fancyfoot[L]{\theauthor}
  \setlength{\headheight}{15pt}
  \setlength{\headsep}{15pt}
}
\pagestyle{fancyplain}

\fancypagestyle{plain}{
  \fancyhf{}
  \fancyfoot[R]{\thepage}
  \setlength{\headheight}{0pt}
  \renewcommand{\headrulewidth}{0pt}
}

\usepackage{listings}
\usepackage{courier}
\usepackage{bashful}

\lstset{
  basicstyle=\ttfamily,
  showstringspaces=false,
  nolol=true
}

\lstdefinestyle{drmCodeStyle}{frame=lines}

\newcommand{\drmcode}[3]{\begin{minipage}{\columnwidth}\lstinputlisting[language={#1},style=drmCodeStyle,caption={#2},name={},nolol=false]{#3}\end{minipage}}
\newcommand{\ccode}[2]{\drmcode{C}{#1}{#2.c}}
\newcommand{\perlcode}[2]{\drmcode{Perl}{#1}{#2.pl}}
\newcommand{\makecode}[2]{\drmcode{Make}{#1}{#2}}

\newenvironment{inlinec}
{ \begin{minipage}[columnwidth]\begin{lstlisting}[language=C] }
{ \end{lstlisting}\end{minipage} }

\newcommand{\kw}[1]{\texttt{#1}}
\newcommand{\key}[1]{\texttt{#1}}
\newcommand{\cfile}[1]{\texttt{#1}}
\newcommand{\tool}[1]{\texttt{#1}}

\begin{document}

\maketitle

\tableofcontents

\lstlistoflistings

\section{Introduction}

This document is an introduction to programming using the C language
and ecosystem. It assumes that the reader is previously familiar with
one or more programming languages, and will not appreciate yet another
introduction to \texttt{if} statements and the strange and wonderful
concept of a `variable'.

\section{Terminal Interaction}

We will make extensive use of a \textsc{Unix} or *NIX
shell. \texttt{Bash} is used throughout; it is assumed that anyone
running anything more exotic is capable of dealing with the
differences.

Shell commands and output are presented as follows:

\bash[script,stdout]
echo "Hello"
\END

\noindent
In other words, an instruction to \emph{run} a command is presented
as:

\bash[script]
echo "Hello"
\END

\noindent
note that the \texttt{\%} sign signifies a shell prompt, and should
not be typed into the terminal. Command output is presented as:

\bash[stdout]
echo "Hello"
\END

\noindent
In almost all cases, 

\section{Hello World}

We should begin by confirming that the C programming environment on
your machine is reasonably sane. For this, we use the traditional
``Hello, World'' program. Enter the following program into the text
editor of your choice, and save it as \cfile{hello.c}.

\ccode{\cfile{hello.c}}{hello}

\noindent
Enter the following command into your terminal to compile
the program:

\bash[script,stdout]
gcc -o hello hello.c
\END

\noindent
and finally run the program with:

\bash[script, stdout]
./hello
\END

\noindent{}
If you get error messages, or the output does not appear as shown,
there is little point continuing until these issues have been
resolved. It is assumed that anyone reading this document has
sufficient intelligence and network bandwidth to resolve problems here
on their own.

\section{The Compilation Process}



\subsection{Preprocessing}

To combine modules at the source level, C uses a preprocessor (often
called, somewhat confusingly, \kw{CPP}). This preprocessor is
\emph{purely textual} --- that is, it works on characters, lines and
files, and has essentially no understanding of a C source file.



Generally, if there are two roughly-equivalent ways of doing something
and one uses the preprocessor, use the other method.

\subsubsection{\kw{#include}}

The most common (or inevitable) use of the preprocessor is to
`include' other files into a source file. We saw this in \cfile{hello.c}
--- the \verb!#include <stdio.h>! line told the preprocessor:
``replace this line with the contents of the file \cfile{stdio.h}, from
the system search path''

First, let's see what happens if we omit this line:

\ccode{\cfile{hellonoinc.c}}{hellonoinc}

\bash[script,stdout]
gcc -o hellonoinc hellonoinc.c
\END

\noindent
This complains about an ``incompatible implicit declaration of
\kw{printf}''. What is happening here? A couple of things.

The first is that printf is declared in \cfile{stdio.h}, so removing the
\verb!#include! line has removed the declaration. A \emph{declaration}
is a line like:

\begin{inlinec}
  int main(int argc, char *argv[])
\end{inlinec}

\noindent
This tells the compiler: ``there is a function \kw{main}, which
returns an \kw{int}, and which takes as arguments one \kw{int} and one
array of \kw{char} pointers''.

A declaration can appear on its own, in which case it must be followed
by a semicolon:

\begin{inlinec}
  int main(int argc, char *argv[]);
\end{inlinec}

\noindent
In addition, any \emph{definition} is also a declaration:

\begin{inlinec}
  int main(int argc, char *argv[]);
  {
    return 0;
  }
\end{inlinec}

\noindent
You can \emph{declare} something as many times as you like (so long as
all the definitions agree) but you may only \emph{define} it
once. Don't worry too much about this now --- it will make more sense
when we talk about modules.

Back to our \verb!#include!-less program. What is happening? The
compiler has no declaration for \kw{printf}, and yet it is compiling
the file anyway! You can run the file with:

\bash[script,stdout]
./hellonoinc
\END

\noindent
It runs perfectly!

What is actually going on here is that the compiler has an `extra'
declaration for \kw{printf}, because it's part of the standard
library, and is in fact handled by a special case. However, it is
treated like a normal C function (and there is no reason it could not
be one), and so it needs a declaration.

The other thing that is going on here is that there is an old
backwards-compatability rule in the C standard that functions which
are referenced without having been declared are give an
\emph{implicit} declaration, which always has the signature:

\begin{lstlisting}[language=C,escapeinside={\%\{}{\%\}}]
  int %{name%}();
\end{inlinec}

\noindent
That is, a function taking no arguments and returning an \kw{int}.

When we try to compile our file, this implicit declaration gets
generated \emph{before} the compiler notices that it already knows
what \kw{printf} is; then the implicit and special declarations
collide, which is what generates the warning.

You might think that having two completely incompatible ideas of what
a function should be is grounds for more than a `warning'. In this
case, there were no ill effects, but it is very possible for this kind
of thing to produce very strange and hard-to-track down bugs. In this
case, we saw the warning and can fix it; when are compiling dozens of
source files into one program, it is easy to miss warnings like this.

For this reason, I (and most other C programmers) prefer to enforce a
``zero warnings'' rule - that is, code must compile cleanly with no
warnings. This is almost always possible; in the cases where it isn't
(because you have a legitimate reason to do something the compiler
doesn't like) then you turn off that specific warning for that specific
case.

We can get the compiler to enforce our ``zero warnings'' rule by
adding the \kw{-Werror} flag to the compiler command line:

\bash[script,stdout]
gcc -Werror -o hellonoinc hellonoinc.c
\END

\noindent
Now the compiler refuses to proceed until we deal with the warning.

While we're at it, we might as well add in a couple more flags that
turn on additional warnings, so that the compiler will tell us about
other dodgy constructs. These flags are \kw{-Wall} and \kw{-Wextra}
(for backwards compatability reasons, \kw{-Wall} does not
\emph{actually} turn on \emph{all} warnings).

Our full command line (for our original program) now looks like:

\bash[script,stdout]
gcc -Wall -Wextra -Werror -o hello hello.c
\END

\noindent
That's getting to be quite a lot of typing. In a little while (in
Section \ref{sec:make}) we will look at a tool that saves us all of
this typing, as well as making it easy to remove specific warnings for
particular files, and saving us lots of time as well.

For now, however, those new to Linux will be pleased to learn that
\kw{bash} has a command history feature. You can go backwards and
forwards in this history using the \key{Up} and \key{Down} arrow keys;
for commands you executed a while ago, you can use \key{Ctrl}+\key{R}
to search incrementally backwards through the command history (each
time you press \key{Ctrl}+\key{R}, it will go backwards to the next
previous match).

Anyway, back to \verb!#include!. You can see what the \verb!#include!
  line is actually doing using the \kw{-E} option to \tool{gcc}. This
  tells \tool{gcc} to stop after the preprocessing stage; if you don't
  provide a \kw{-o} option, the preprocessed file will be sent to
  \kw{STDOUT} (in this case, your terminal).

Run the command:

\bash[script]
gcc -E hello.c
\END

\noindent
This will produce an enormous amout of guff, most of which is quite
inscrutable. \cfile{stdio.h} pulls in a number of other files to
provide particular types and other features it depends on (if you're
interested, you can probably find it at \cfile{/usr/include/stdio.h}).

However, if we \tool{grep} the output, we can find what we're looking
for:

\bash[script,stdout]
gcc -E hello.c | grep printf
\END

\noindent
This list contains a few other \kw{printf}-ish functions, and the line
for our \emph{invocation} of printf, but you can clearly see that the
file now contains the (correct) definition of \kw{printf}.

\subsubsection{Conditional Compilation}

\subsubsection{Macros}

\subsection{Compilation}

\subsection{Linking}

\subsection{Static Libraries}

\subsection{\tool{make}}
\label{sec:make}

\section{The C Memory Model}

\subsection{Pointers}

\subsection{Arrays}

\subsection{Allocation}

\section{Debugging}

\section{Modules}

There are a number of other ways to create modules in C (the advantage
of having to do everything manually is that there can be a lot of
flexibility in how you do it). However, these two are the most common,
so we will look at these first.

\subsection{Single-Instance Modules}

For both of our module examples, we will be recreating the
standard-library function \kw{strtok}. \kw{strtok} provides a (fairly
limited) mechanism for converting strings into tokens.

\kw{strtok} is called like this:

\begin{inlinec}
  char *str = "These are some words";
  char *tok = strtok(str, " ");
\end{inlinec}

This will cause \kw{tok} to point to a string containing
``These''. After the first invocation, we can call \kw{strtok} like
this:

\begin{inlinec}
  tok = strtok(NULL, " ");
\end{inlinec}

and \kw{strtok} will continue tokenising the same string (you have to
keep providing the delimiters).

If, instead, we had done this:

\begin{inlinec}
  char *str = "length,width,height;roll,pitch,yaw";
  char *tok = strtok(str, ",;");
\end{inlinec}

And kept calling \kw{strtok} with \kw{NULL} and the same delimiters,
we would get, in turn, the strings ``length'', ``width'', ``height'',
``roll'', ``pitch'', ``yaw''.

Roughly, therefore, what \kw{strtok} does is split the its first
argument on \kw{any} of the characters in its second argument.

\subsection{Multiple-Instance Modules}

\end{document}
